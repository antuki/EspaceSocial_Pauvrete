% This is the Reed College LaTeX thesis template. Most of the work
% for the document class was done by Sam Noble (SN), as well as this
% template. Later comments etc. by Ben Salzberg (BTS). Additional
% restructuring and APA support by Jess Youngberg (JY).
% Your comments and suggestions are more than welcome; please email
% them to cus@reed.edu
%
% See https://www.reed.edu/cis/help/LaTeX/index.html for help. There are a
% great bunch of help pages there, with notes on
% getting started, bibtex, etc. Go there and read it if you're not
% already familiar with LaTeX.
%
% Any line that starts with a percent symbol is a comment.
% They won't show up in the document, and are useful for notes
% to yourself and explaining commands.
% Commenting also removes a line from the document;
% very handy for troubleshooting problems. -BTS

% As far as I know, this follows the requirements laid out in
% the 2002-2003 Senior Handbook. Ask a librarian to check the
% document before binding. -SN

%%
%% Preamble
%%
\pdfobjcompresslevel 0 % pour être pris en charge par la plateforme d'archivage du CINES
% voir https://facile.cines.fr#latex
% \documentclass{<something>} must begin each LaTeX document
\documentclass[12pt,a4paper]{reedthesis}
% Packages are extensions to the basic LaTeX functions. Whatever you
% want to typeset, there is probably a package out there for it.
% Chemistry (chemtex), screenplays, you name it.
% Check out CTAN to see: https://www.ctan.org/
%%
\usepackage{graphicx,latexsym}
\usepackage{amsmath,amssymb,amsthm}
\usepackage{amsfonts}
\usepackage{longtable,booktabs,setspace}
% \usepackage{chemarr} %% Useful for one reaction arrow, useless if you're not a chem major
\usepackage[hyphens]{url}
% Added by CII
\usepackage{lmodern}
\usepackage{float}
\floatplacement{figure}{H}
% End of CII addition
\usepackage{rotating}

\usepackage[utf8]{inputenc}
\usepackage[T1]{fontenc}
\usepackage{fancyhdr}
\usepackage{xcolor}
\definecolor{Prune}{RGB}{99,0,60}
\usepackage{mdframed}
\usepackage{multirow} %% Pour mettre un texte sur plusieurs rangées
\usepackage{multicol} %% Pour mettre un texte sur plusieurs colonnes
\usepackage{scrextend} %Forcer la 4eme  de couverture en page pair
\usepackage{tikz}
\usepackage[absolute]{textpos}
\usepackage{colortbl}
\usepackage{array}
%\RequirePackage{geometry}% That nicely create a one-page template
%\geometry{textheight=100ex,textwidth=40em,top=30pt,headheight=30pt,headsep=30pt,inner=80pt}
\usepackage{geometry}
\usepackage{hyperref}

% Next line commented out by CII
%%% \usepackage{natbib}
% Comment out the natbib line above and uncomment the following two lines to use the new
% biblatex-chicago style, for Chicago A. Also make some changes at the end where the
% bibliography is included.
%\usepackage{biblatex-chicago}
%\bibliography{thesis}


% Added by CII (Thanks, Hadley!)
% Use ref for internal links
\renewcommand{\hyperref}[2][???]{\autoref{#1}}
\def\chapterautorefname{Chapitre}
\def\sectionautorefname{Section}
\def\subsectionautorefname{Sous-section}
% End of CII addition

% Added by CII
\usepackage{caption}
\captionsetup{width=5in}
% End of CII addition

% \usepackage{times} % other fonts are available like times, bookman, charter, palatino

% Syntax highlighting #22


% Added by CII
%%% Copied from knitr
%% maxwidth is the original width if it's less than linewidth
%% otherwise use linewidth (to make sure the graphics do not exceed the margin)
\makeatletter
\def\maxwidth{ %
  \ifdim\Gin@nat@width>\linewidth
    \linewidth
  \else
    \Gin@nat@width
  \fi
}
\makeatother

%Added by @MyKo101, code provided by @GerbrichFerdinands
\newlength{\cslhangindent}
\setlength{\cslhangindent}{1.5em}
\newenvironment{CSLReferences}%
  {}%
  {\par}

\renewcommand{\contentsname}{Sommaire}
\renewcommand{\listfigurename}{Listes des figures}
\renewcommand{\listtablename}{Listes des tableaux}
\renewcommand\chaptername{Chapitre}
% \renewcommand\appendixtocname}{Annexes}
% \renewcommand\appendixpagename}{Annexes}
\renewcommand{\appendixname}{Annexe}

% End of CII addition

\setlength{\parskip}{0pt}

% Added by CII

\providecommand{\tightlist}{%
  \setlength{\itemsep}{0pt}\setlength{\parskip}{0pt}}

% End of CII addition
%%
%% End Preamble
%%
%
\begin{document}
\begin{titlepage}

%\thispagestyle{empty}

\newgeometry{left=7.5cm,bottom=2cm, top=1cm, right=1cm}
\tikz[remember picture,overlay] \node[opacity=1,inner sep=0pt] at (-28mm,-135mm){\includegraphics{logos/bandeau.pdf}};

% fonte sans empattement pour la page de titre
\fontfamily{fvs}\fontseries{m}\selectfont

%*****************************************************
%******** NUMÉRO D'ORDRE DE LA THÈSE À COMPLÉTER  ****
%******** après le premier dépôt légal /          ****
%******** French legal PhD number to be completed ****
%******** after the first legal deposit           ****
%*****************************************************

\color{white}
\begin{picture}(0,0)
\put(-150,-735){\rotatebox{90}{Année scolaire~: 2020-2021}}
\end{picture}
%*************************************************************
%**  LOGO  ÉTABLISSEMENT PARTENAIRE SI COTUTELLE :          **
%**  CHANGER L'IMAGE logoCotutelle.png; SINON COMMENTER  /  **
%**  Logo of partner establishment if cotutelle agreement : **
%**  change image logoCotutelle.png; otherwise add % signs  **
%*************************************************************
\vspace{10mm}
\vspace{-20mm} % à ajuster en fonction de la hauteur du logo
\flushright \includegraphics[width=3cm]{logos/logo.png}

%*****************************************************
%**************** TITRE / TITLE **********************
%*****************************************************
\flushright
% \vspace{15mm} % largeur à régler éventuellement / width to adjust if necessary
\vspace{0mm} 
\color{Prune}
\fontfamily{fvs}\fontseries{m}\fontsize{22}{26}\selectfont

  
Construire l'espace social 

\medskip

de la pauvreté avec un 

\medskip

Baromètre d'opinion


%*****************************************************

%\fontfamily{fvs}\fontseries{m}\fontsize{8}{12}\selectfont
\normalsize
% \vspace{15mm}
\vspace{10mm}

\color{black}
\large
\textbf{Sociologie Quantitative \& Démographie}
\normalsize

% \hspace*{-0.7cm}\\
% \small Spécialité de doctorat~: \\
% \footnotesize Unité de recherche~: \\
% \footnotesize Référent~: 

\vspace{10mm}
\begin{center}
\includegraphics[height=8cm]{logos/accueil.png}
\end{center}
\vspace{10mm}

\textbf{Mémoire présenté et soutenu à Paris}

\medskip

\textbf{En septembre 2021, par}

\bigskip
\bigskip

\Large {\color{Prune} \textbf{Kim ANTUNEZ}}

\vspace{15mm}

\flushleft \normalsize \textbf{Composition du jury~:}
\bigskip

\scriptsize
\arrayrulecolor{Prune}
\begin{tabular}{|p{4cm}l}
\textbf{Ivaylo Petev} &  Enseignant-chercheur en sociologie (CREST, ENSAE)\\
 & \\
\textbf{Nicolas Robette} &  Enseignant-chercheur en sociologie (CREST, ENSAE)\\
 & \\
% \textbf{} &  \\
%  & \\
% \textbf{} &  \\
%  & \\
% \textbf{} &  \\
%  & \\
% \textbf{} &  \\
%  & \\
\end{tabular}
% \begin{tabular}{p{8cm}l}
% & \\
% \textbf{} &  \\
%  & \\
% \textbf{} &  \\
%  & \\
% \textbf{} &  \\
%  & \\
% \end{tabular}

\end{titlepage}
\addamargin % this add the margin back

\frontmatter % this stuff will be roman-numbered
\pagestyle{empty} % this removes page numbers from the frontmatter
  \begin{acknowledgements}
    Ecrire la pr?face
  \end{acknowledgements}
  \begin{abstract}
    Résumé à écrire.

    \par

    Ici.
  \end{abstract}
% 
  \hypersetup{linkcolor=black}
  \setcounter{secnumdepth}{2}
  \setcounter{tocdepth}{2}
  \tableofcontents




\mainmatter % here the regular arabic numbering starts
\pagestyle{fancyplain} % turns page numbering back on

\hypertarget{introduction}{%
\chapter*{Introduction}\label{introduction}}
\addcontentsline{toc}{chapter}{Introduction}

Introduction\ldots{} Introduction\ldots{}

Intro !

\hypertarget{rmd-basics}{%
\chapter{Titre}\label{rmd-basics}}

Début du chapitre 1.

\hypertarget{conclusion}{%
\chapter*{Conclusion}\label{conclusion}}
\addcontentsline{toc}{chapter}{Conclusion}

\appendix

\hypertarget{titre}{%
\chapter{Titre}\label{titre}}

\backmatter

\hypertarget{bibliographie}{%
\chapter*{Bibliographie}\label{bibliographie}}
\addcontentsline{toc}{chapter}{Bibliographie}

\markboth{Bibliographie}{Bibliographie}

\noindent

\setlength{\parindent}{-0.20in}
\setlength{\leftskip}{0.20in}
\setlength{\parskip}{8pt}

\hypertarget{refs}{}
\begin{CSLReferences}{1}{0}
\leavevmode\hypertarget{ref-angel2000}{}%
Angel, E. (2000). \emph{Interactive computer graphics : A top-down approach with OpenGL}. Boston, MA: Addison Wesley Longman.

\leavevmode\hypertarget{ref-angel2001}{}%
Angel, E. (2001a). \emph{Batch-file computer graphics : A bottom-up approach with QuickTime}. Boston, MA: Wesley Addison Longman.

\leavevmode\hypertarget{ref-angel2002a}{}%
Angel, E. (2001b). \emph{Test second book by angel}. Boston, MA: Wesley Addison Longman.

\end{CSLReferences}

% 4eme de couverture
% \ifthispageodd{}{\newpage\thispagestyle{empty}\null}
% Use the next command with TeX Live version ≥ 2020
%\Ifthispageodd{}{\newpage\thispagestyle{empty}\null}
\newpage
\thispagestyle{empty}
\newgeometry{top=1.5cm, bottom=1.25cm, left=2cm, right=2cm}
\fontfamily{rm}\selectfont

\lhead{}
\rhead{}
\rfoot{}
\cfoot{}
\lfoot{}

\noindent
%*****************************************************
%***** LOGO DE L'EDMH *********
%*****************************************************
% \includegraphics[height=4.5cm]{}
% \vspace{1cm}
%*****************************************************
\begin{mdframed}[linecolor=Prune,linewidth=1]
\vspace{-.25cm}
\paragraph*{Titre~:} Construire l'espace social de la pauvreté avec un Baromètre d'opinion
\begin{small}
\vspace{-.25cm}
\paragraph*{Mots clés~:} Mots-clefs en français.

\vspace{-.5cm}
\begin{multicols}{2}
\paragraph*{Résumé~:} Résumé en français
\end{multicols}
\end{small}
\end{mdframed}
\begin{mdframed}[linecolor=Prune,linewidth=1]
\vspace{-.25cm}
\paragraph*{Title:} Titre anglais
\begin{small}
\vspace{-.25cm}
\paragraph*{Keywords:} Keywords in English.

\vspace{-.5cm}
\begin{multicols}{2}
\paragraph*{Abstract:} Abstract in English
\end{multicols}
\end{small}
\end{mdframed}

\vfill
\fontfamily{fvs}\fontseries{m}\selectfont
\noindent\begin{tabular}{p{14cm}}
\multirow{3}{16cm}[+0mm]{\small {\color{Prune} {\bf Université Paris-Saclay}\\
{\scriptsize Espace Technologique / Immeuble Discovery}\\
{\scriptsize  Route de l’Orme aux Merisiers RD 128 / 91190 Saint-Aubin, France}}}\\\mbox{}
\end{tabular}

% Index?

\end{document}
